\documentclass[mathserif, 10pt, aspectratio=169]{beamer}
\usepackage[utf8]{inputenc}
\usepackage{amsmath, amsfonts}
\usepackage{appendixnumberbeamer}


\title{Light New Physics coupling to $\tau$}
\subtitle{{\bf JA}, G. Levati, P. Paradisi, S. Rigolin, N. Selimovic}
\author[Jorge Alda]{Jorge Alda \hspace{4em} \texttt{jorge.alda@pd.infn.it} \\
Università degli Studi di Padova \& CAPA}
\date[Saturnalia '23]{Saturnalia '23 \\ 21-12-2023}



\usetheme{Zaragoza}
\usecolortheme{Unipd}
\titlepagelogoA{\includegraphics[width=4cm]{logos/CAPA.png}}
\titlepagelogoB{\includegraphics[width=6cm]{logos/unipd.png}}




\begin{document}
\begin{frame}[noframenumbering,plain]

\titlepage

\end{frame}

\begin{frame}\frametitle{Why New Physics?}
    Our objective is to look for signs of New Physics, motivated by\vspace{20pt}
    \begin{itemize}\setlength{\itemsep}{10pt}
        \item Theoretical questions: Flavour puzzle, dark matter, dark energy, unification with gravity, hierarchy problem, etc.
        \item Experimental anomalies: $(g-2)_\mu$, $R_{D^{(*)}}$, Cabbibo anomaly, etc.
        \item Elaboration of hypotheses.
    \end{itemize}
\end{frame}

\begin{frame}\frametitle{Why Light New Physics?}
    \begin{itemize}\setlength{\itemsep}{10pt}
        \item We could see it in the current particle colliders in the form of resonances (``visible'' decays) or missing energy (``invisible'' decays).
        \item Also in other experiments: helioscopes, astronomical observations, etc.
        \item Can not be described as an Effective Field Theory.
        \item Theoretical motivation: Dark Matter candidates, Strong CP problem, axiverse.
    \end{itemize}
\end{frame}

\begin{frame}\frametitle{Strong CP problem}
    \begin{itemize}\setlength{\itemsep}{10pt}
        \item Three discrete transformations: Charge conjugation (C), Parity (P) and Time reversal (T).
        \item Experimentally, C, P and CP are not symmetries of the SM.
        \item Strong interactions preserve CP, although we could write a CP-violating term $\theta G\tilde{G}$.
        \item Very strong experimental bounds from electric dipole moment of the neutron.
        \item {\bf Peccei-Quinn mechanism:} A new pseudo-scalar field with anomalous couplings to gluons $a G\tilde{G}$ which develops a vev, dynamically erasing the CP violation. Its particle excitation is the axion.
        \item Characterized by energy scale $f_a$ and mass $m_a f_a \sim m_\pi f_\pi$.
        \item Shift symmetry $a \to a + \mathrm{constant}$.
    \end{itemize}
\end{frame}

\begin{frame}\frametitle{Axion-Like Particles}
    \begin{itemize}\setlength{\itemsep}{15pt}
        \item Many beyond-SM models propose a new global $U(1)$ symmetry, spontaneously broken at energies $f_a \gg v$. The Nambu-Goldstone boson (NGB) associated to this symmetry would be an Axion-like particle (ALP).
        \item If the symmetry is also explicitly broken, the ALP is a pseudo-NGB, and $m_a f_a \nsim m_\pi f_\pi$.
        \item As an example, string theory predicts the existence of many ALPs in a wide range of masses and energy scales as a result of the compactification of antisymmetric tensor fields.
    \end{itemize}
\end{frame}

\begin{frame}\frametitle{Why coupling to $\tau$?}
    \begin{columns}
        \begin{column}{0.6\textwidth}
            \begin{itemize}\setlength{\itemsep}{10pt}
                \item Many experimental constraints for couplings to photons and to quarks.
                \item The couplings to fermions are proportional to their mass, and $\tau$ is the heaviest lepton.
                \item New Physics in 3rd generation, consistent under RG flow.
                \item Improved experimental sensitivity to $\tau$ (e.g in Belle-II).
            \end{itemize}
        \end{column}
        \begin{column}{0.27\textwidth}
            \includegraphics[width=\columnwidth]{figures/Neubert_Ce_Bounds.pdf}\\{\scriptsize A. Biekötter, J. Fuentes-Martín, A. M. Galda and M. Neubert, arXiv:2307.10372}
        \end{column}
    \end{columns}
\end{frame}

\begin{frame}\frametitle{Light New Physics}
    Axion-like Particle coupled to a Peccei-Quinn current of leptons

    $$\mathcal{L}_\mathrm{ALP} = \frac{1}{2}\partial_\mu a \partial^\mu a - \frac{1}{2} m_a^2 a^2 - \frac{1}{2 f_a}\partial_\mu a j^\mu_\mathrm{PQ}\,;$$

    $$j^\mu_\mathrm{PQ} = \sum_{i,j} \left( c_\ell^{ij} \bar{\ell}_i\gamma^\mu \gamma_5 \ell_j + \bar{c}_\ell^{ij} \bar{\ell}_i\gamma^\mu  \ell_j  + c_\nu^{ij} \bar{\nu}_{\ell_i} \gamma^\mu P_L \nu_{\ell_j} \right)\,. $$

    \begin{itemize}
        \item $m_a \in [1\,\mathrm{MeV}, 10\,\mathrm{GeV}]$, $f_a \sim 1\,\mathrm{TeV}$, flavour-universal $c^{ij} = c \delta^{ij}$.
        
        \item $g_\ell = c_\ell m_\ell/f_a$.
        
        \item After integration-by-parts and equations-of-motion
        $$\mathcal{L}_\mathrm{ALP, int} = \sum_\ell \left(i g_\ell \bar{\ell}\gamma_5\ell a + \frac{ig}{2 \sqrt{2} m_\ell} (g_\ell - \bar{g}_\ell + g_{\nu_\ell}) (\bar{\ell}\gamma^\mu P_L \nu_\ell) W^-_\mu a + \mathrm{h.c.} \right) + (V\tilde{V}a)\,.$$

        \item Electroweak-preserving case: $g_\ell - \bar{g}_\ell + g_{\nu_\ell}=0$.
    \end{itemize}
\end{frame}

\begin{frame}\frametitle{Light New Physics}

    Scalar $\phi$ and pseudo-scalar $\hat{\phi}$ bosons:

    $$\mathcal{L}_\mathrm{light NP} \subset \frac{1}{2}\partial_\mu \phi \partial^\mu \phi - \frac{1}{2} m_\phi^2 \phi^2 + \frac{1}{2}\partial_\mu \hat{\phi} \partial^\mu \hat{\phi} - \frac{1}{2} m_{\hat{\phi}}^2 \hat{\phi}^2 - \sum_\ell \bar{\ell}(k_\ell \phi + i \hat{k}_\ell \hat{\phi}\gamma_5) \ell\,.$$

    ~
    
    For the pseudo-scalar boson, we recover the EW-preserving ALP when the couplings are hierarchical $\hat{k}_\ell = g_\ell = c m_\ell/f_a$.
\end{frame}

\begin{frame}\frametitle{Particle decays}
    
    \begin{columns}

        \begin{column}{0.6\textwidth}
            The NP particles can decay to a pair of leptons

            $$\Gamma(S \to \ell^+\ell^-) = \frac{m_S}{8\pi} |K_\ell|^2 \left(1-\frac{4 m_\ell^2}{m_S^2}\right)^{\alpha_S}\,,$$

            {\small with $K_\ell = g_\ell$ and $\alpha_S = 1/2$ for $S=a$, and $K_\ell = k_\ell$ and $\alpha_S = 3/2$ for $S=\phi$.}

            Also decays to $2\gamma$ through a lepton loop.

            ~

            ALPs with $m_a > 2 m_e$ and scalars will typically decay inside the detector.
        \end{column}
        \begin{column}{0.38\textwidth}
            \includegraphics[width=\columnwidth]{figures/BR_S.png} \\
            \includegraphics[width=\columnwidth]{figures/properlength.png}
        \end{column}
    \end{columns}
\end{frame}



\end{document}